\section{Cel Projektu}
Celem projektu jest demonstracja możliwości zrównoleglenia obliczeń Szybkiej Transformaty Fouriera (FFT), na przykładzie przetwarzania obrazów.
\section{Wstęp}
\subsection{Technologia i zakres projektu}
\subsection{Wprowadzenie do tematyki FFT}
\section{Opis rozwiązania}
\section{Szczegóły instalacji i uruchomienia programu}
\section{Uzyskane rezultaty}
\section{Wnioski}
\begin{itemize}
	\item Po zastosowaniu transformaty w obie strony na obrazku zauważalny jest drobny spadek jakości obrazu.
	\item Zrównoleglenie na procesy poszeczególnych grafik dało bardzo dobre rezultaty i zdecydowanie przyśpieszyło działanie programu przy większej ilości obrazków.
	\item Wersja sekwencyjna jest równie szybka a czasami szybsza dla 2 lub 3 obrazków, ale od 4 i więcej zauważalna jest tendencja przyrostowa czasu wykonywania transformaty. W przeciwieństwie do zrównoleglonej wersji, gdzie czas utrzymuje się mniej więcej na tym samym poziomie.
	\item Użycie zbyt dużej ilości wątków wbrew pozorom daje gorsze rezultaty niż ich mniejsza ilość, np. 8 czy 4. Jest to spowodowane, że program traci czas na uruchomienie poszczególnych wątków, a one same w sobie nie mają dużego obciążenia obliczeniowego
\end{itemize}

