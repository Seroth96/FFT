\section{Cel Projektu}
Celem projektu jest demonstracja możliwości zrównoleglenia obliczeń Szybkiej Transformaty Fouriera (FFT), na przykładzie przetwarzania obrazów.
\section{Wstęp}
\subsection{Technologia i zakres projektu}
\subsection{Wprowadzenie do tematyki FFT}
\section{Opis rozwiązania}
\section{Szczegóły instalacji i uruchomienia programu}
Po otrzymaniu pliku wykonywalnego \textbf{.exe} należy umieścić go w folderze, w którym znajduje się folder \textbf{img} a w nim obrazy, na których chcemy wykonać FFT. Obrazki muszą być w formacie \textbf{.bmp} oraz w zapisie 8bpp. 

Aby uruchomić program należy:
\begin{enumerate}
	\item Otworzyć \textbf{Wiersz poleceń (cmd)}
	\item Przejść do folderu z plikiem \textbf{.exe}
	\item Umieścić obrazy w folderze \textbf{img}, jeśli go nie ma należy taki utworzyć
	\item Uruchomić program z konsoli poleceniem:\\
	 \textbf{\textit{FFT.exe [nazwa\_obrazka1] [nazwa\_obrazka1] [nazwa\_obrazka2] [nazwa\_obrazka3] [nazwa\_obrazka4]}}
	 \subitem [nazwa\_obrazka] - to jest nazwa naszego obrazu bez rozszerzenia \textbf{.bmp}
	 \subitem FFT.exe - zależy od tego czy zmieniliśmy nazwę pliku wykonywalnego, podana nazwa jest domyślna
	\item Dla każdego podane obrazka utworzył się w folderze \textbf{img} obrazek po FFT oraz po odwróconej FFT. (do nazw dodane przyrostki \textit{fourier} albo \textit{backwardfourier})
\end{enumerate} 
\clearpage\newpage
\section{Uzyskane rezultaty}
\begin{figure}[ht]
	\includegraphics[width=\textwidth]{figures/Seq13.png}
	\centering
	\caption{Dla 13 obrazków 512x512 pikseli - wersja sekwencyjna}
\end{figure}
\begin{figure}[ht]
	\includegraphics[width=0.8\textwidth]{figures/Par13Wat4.png}
	\centering
	\caption{Dla 13 obrazków 512x512 pikseli - wersja zrównoleglona 4 wątki}
\end{figure}
\begin{figure}[ht]
	\includegraphics[width=\textwidth]{figures/Par13Wat8.png}
	\centering
	\caption{Dla 13 obrazków 512x512 pikseli - wersja zrównoleglona 8 wątków}
\end{figure}
\begin{figure}[ht]
	\includegraphics[width=\textwidth]{figures/Par13Wat16.png}
	\centering
	\caption{Dla 13 obrazków 512x512 pikseli - wersja zrównoleglona 16 wątków}
\end{figure}

\begin{figure}[ht]
	\includegraphics[width=\textwidth]{figures/Seq4.png}
	\centering
	\caption{Dla 4 obrazków 512x512 pikseli - wersja sekwencyjna}
\end{figure}
\begin{figure}[ht]
	\includegraphics[width=\textwidth]{figures/Par4Wat8.png}
	\centering
	\caption{Dla 4 obrazków 512x512 pikseli - wersja zrównoleglona 8 wątków}
\end{figure}

\begin{figure}[ht]
	\includegraphics[width=\textwidth]{figures/Seq2.png}
	\centering
	\caption{Dla 2 obrazków 512x512 pikseli - wersja sekwencyjna}
\end{figure}
\begin{figure}[ht]
	\includegraphics[width=\textwidth]{figures/Par2Wat8.png}
	\centering
	\caption{Dla 2 obrazków 512x512 pikseli - wersja zrównoleglona 8 wątków}
\end{figure}

\clearpage\newpage
\section{Wnioski}
\begin{itemize}
	\item Po zastosowaniu transformaty w obie strony na obrazku zauważalny jest drobny spadek jakości obrazu.
	\item Zrównoleglenie na procesy poszeczególnych grafik dało bardzo dobre rezultaty i zdecydowanie przyśpieszyło działanie programu przy większej ilości obrazków.
	\item Wersja sekwencyjna jest równie szybka a czasami szybsza dla 2 lub 3 obrazków, ale od 4 i więcej zauważalna jest tendencja przyrostowa czasu wykonywania transformaty. W przeciwieństwie do zrównoleglonej wersji, gdzie czas utrzymuje się mniej więcej na tym samym poziomie.
	\item Użycie zbyt dużej ilości wątków wbrew pozorom daje gorsze rezultaty niż ich mniejsza ilość, np. 8 czy 4. Jest to spowodowane, że program traci czas na uruchomienie poszczególnych wątków, a one same w sobie nie mają dużego obciążenia obliczeniowego
\end{itemize}

